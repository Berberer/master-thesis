% !TEX root = ../my-thesis.tex
%
\chapter{Implementation}
\label{sec:implementation}
For the approach of this thesis, describes in the previous chapter, a reference implementation in the \textit{Python} programming language is produced, which can be found here:~\url{https://github.com/Berberer/frankensteins-automl}.

This reference implementation serves two different reason:
\begin{itemize}
	\item Help for a better understanding as an alternative and more practical representation of the approach, since it can be easier to read structured code instead of a theoretical and lengthy formal definition.
	\item For the evaluation of this approach and the comparison against other approaches and some baseline values in chapter~\ref{sec:evaluation}, this approach must be provided as an executable program to get results for different experimental settings.
\end{itemize}

The implementation is explained in the following chapter, which is structured in three parts.
At first, the architectural composition of the implementation and the interaction of the included components is explained.
With this overall perspective of the implementation, the codebase will be easier to understand and to navigate.
Since the implementation was not done from scratch, the utilized Python libraries are listed and their functionality explained as a second part to pay respect to their creators as well as elucidate their contribution to the overall functionality.
Finally, the description schema of the AutoML optimization space definition is provided, which is a required input for the implementation, such that a user can understand the included optimization spaces of the implementation and modify them if needed.

\section{Components of the Project and their Interaction}
\label{sec:implementation:components}

\Blindtext

\section{Utilized Python Libraries}
\label{sec:implementation:libraries}

\Blindtext

\section{Exemplaric File Format for defining the AutoML Optimization Space}
\label{sec:implementation:json}

\Blindtext
