% !TEX root = ../my-thesis.tex
%
\chapter{Introduction}
\label{sec:intro}

\section{Motivation}
\label{sec:intro:motivation}

Since machine learning is present in a fast-growing part of everyday life, economy, and science, the number of people who want to use machine learning and apply it to their use case is equally fast-growing.
The problem is that using machine learning beyond simple examples and tutorials requires theoretical domain knowledge as well as solid programming skills.\newline
Applying a machine learning approach includes the selection of a suitable learning algorithm or even multiple algorithms and other components combined as a complex pipeline.
The cautious choice of machine learning algorithms and other data science components for each problem or use-case is crucial to have a suitable and well-performing pipeline.\newline
Each selected pipeline component may expose hyperparameters.
In turn, this hyperparameter configuration has to be tuned in terms of the specific problem instance.
This a second necessary step for the application of machine learning, because the optimal configuration values differ between problem instances.\newline
Finally, the selected model or pipeline with the configured parameters must be implemented with a suitable programming language within a machine learning framework or toolbox, which are usually comparably big and complex to use.

Therefore, caused by these three necessary steps, there is a learning and knowledge barrier that prevents interested people with a non-technical background from using machine learning.
Without having this deep machine learning domain knowledge, it is very hard to make suitable choices for the machine learning algorithm as well as corresponding hyperparameters.\newline
But when the choices were not adequate, the performance measurements of the machine learning pipeline will only increase for a very high amount of training data or even not increase at all.
However, the amount of data, which can be used to train a machine learning algorithm, is often limited for most use-case domains.
Therewith, inadequate choices for components or configurations can not be compensated with a higher amount of data for these use-cases.
Additionally, since this necessary deep knowledge in machine learning for making adequate choices is not broadly available and a lot of companies or research facilities have a high demand for machine learning applications, it is not possible to apply machine learning in every use-case where it might be beneficial.

In the last years, different approaches were developed, which can empower nearly everyone to use machine learning for datasets of their domain or setting.
These approaches allow this by automating this algorithm selection and configuration problem, consisting of the aforementioned three required sub-tasks: Selecting algorithms (respective components of pipelines), configuring all required hyperparameters, and constructing a usable implementation.
All the end-user has to do is to provide their data in a suitable dataset format and usually specify a resource or time budget for the solver of the combined algorithm selection and configuration problem.\newline
Within this budget, the best pipeline can be searched regarding a metric, as for example predictive accuracy.
Hence, this problem of finding a machine learning pipeline for a dataset can be considered an optimization problem.
Solving this optimization problem is the challenge of a research area called \textit{Automated Machine Learning} or short \textit{AutoML}.

AutoML received much attention in the scientific machine learning community and a broad spectrum of different methods were developed and published.
The majority of these approaches and the plethora of different strategies they utilize have usually one thing in common.
They see the model selection and the model configuration as a joint optimization problem and therewith solve these two problems within a single optimization space with a single optimization algorithm.\newline
However, recently a few new approaches were published that partition this joint optimization space and approach with different optimization methods the two respective sub-spaces alternatingly.
But this application of more than one optimization algorithm can be extended to three or more optimizers and in theory, this should show an improvement because of the \textit{No-free-lunch Theorems for optimization}.
It states that if an optimization algorithm performs under constraints superior for one problem or class of problems it has to pay for this by performing inferior for other problems.
In the context of AutoML, this would imply that one optimization approach, can not yield the best solution across all datasets, timeouts, and pipeline component domains.

Combining multiple optimization methods into an ensemble of optimizers would allow this ensemble to optimize each possible pipeline with the most suitable optimization algorithm with respect to the parameter space of this pipeline and the properties of the dataset.
Therewith, the most suitable optimization algorithm out of the ensemble could be identified and exploited to the given AutoML problem setting.\newline
For creating and utilizing these optimizer ensembles in the context of AutoML, a novel approach is required and is the topic of this thesis. 

\section{Thesis Goal and Research Questions}
\label{sec:intro:goal}
The recent related works approaches increased their number of applied optimizer up to two instead of the single optimizer in the established state-of-the-art approaches.
However, implied by the No-free-lunch Theorems for optimization, this method would achieve the optimal results for only a slightly bigger set of problem instances compared to single optimizer approaches.\newline
If the integrated selection of optimizers would be increased even further, this set of problem instances could theoretically be increased even further, under the assumption that more optimizers than just two are considered and the most suitable one is exploited.
Additionally, during the evaluation of the different optimizers to identify the most suitable one for the problem instance, they could share gathered data between each other about their candidate evaluations up to that point.\newline
This aspect of data sharing as well as identification and exploitation of the best suited optimizer introduces the novel approach of optimizer ensembles for AutoML problems.
In the context of this thesis, a feasible concept and algorithm for such optimizer ensembles are devised and exemplarily realized as a reference implementation.
Afterwards, the implemented optimizer ensemble algorithm is the test subject for a series of empirical evaluations.\newline
Overall, the goal of this thesis is to answer two research questions regarding this approach for optimizer ensembles applied to the AutoML problem on the basis of the data gathered during these empirical evaluations:
\begin{enumerate}
    \item Beyond the theoretical assumptions based on the No-Free-Lunch theorems, is the devised approach for optimizer ensembles indeed feasible in the context of AutoML?
    More precisely, is the result quality comparable or even better compared to other state-of-the-art approaches?
    And if it is a feasible approach, what influence have the particular components of the approach on the overall outcome?
    \item Can knowledge about the optimization in the general AutoML context be extrapolated from this approach?
    From a theoretical point of view, the suitability of an optimization algorithm for a model configuration problem could depend on a variety of different factors like the dataset, the optimization budget, the components of the machine learning pipeline whose parametrizations are optimized, and several others.
    The pipeline components are influenced by the model selection and the timeout depends on the environment of the user.
    But the relationship between input datasets as well as more generalized properties of the datasets in relation to the optimizer utilization can be evaluated and could be valuable insights beyond the discussion approach.\newline
    When the frequencies of utilizing certain optimizations algorithms during the execution of this approach are recorded, this may give indications regarding their capability for different AutoML problems, since this approach tries to find the best suited optimizer out of the ensemble for the input problem, i.e. the dataset with its properties, and exploit it.
    Based on these frequencies, is every optimizer called an equal number of times, or is one or more optimization method favored for the AutoML use case?
    Does it depend on certain dataset properties of the concrete AutoML problem, whether an optimization method is used relatively often?
\end{enumerate}

\section{Thesis Structure}
\label{sec:intro:structure}
Following the current introduction chapter, this thesis is structured into five chapters.
The content and goals of each chapter are briefly explained here:

\textbf{Chapter~\ref{sec:theory}} \\[0.2em]
At first, the foundations of the machine learning, the AutoML setting, and optimization in general are elucidated, to give a starting point for the related work part of this chapter and the approach of this thesis.

\textbf{Chapter~\ref{sec:approach}} \\[0.2em]
The optimizer ensemble approach of this thesis is theoretically specified here and the underlying design choices are explained and justified.

\textbf{Chapter~\ref{sec:implementation}} \\[0.2em]
To apply the approach for actual AutoML problem settings as well as evaluating the approach empirically in an experiment series, an actual implementation is necessary and is explicated in this chapter.

\textbf{Chapter~\ref{sec:evaluation}} \\[0.2em]
In order to answer the previously listed research questions, a series of empirical experiments is designed and the results analyzed in this chapter.

\textbf{Chapter~\ref{sec:conclusion}} \\[0.2em]
Finally, the answers to the research questions are used to summarize the concepts of this approach and their validity.
At last, an outlook is given with a selection of starting points for future work and follow-up research.
