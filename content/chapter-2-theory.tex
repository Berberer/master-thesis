% !TEX root = ../my-thesis.tex
%
\chapter{Theoretical Foundations}
\label{sec:theory}
Before presenting the approach for tackling AutoML with an ensemble of optimizers, some theoretical foundations of both elements, AutoML and optimization, are given in the following.
This theoretical background is structured in three parts:
\begin{enumerate}
    \item Some basic concepts and intuitions of machine learning in general are outlined alongside with the challenges and problems that arise when applying machine learning.
    \item The concepts and usual approaches of AutoML are introduced, which were developed to tackle the listed challenges of classical machine learning. In addition, the connection between the AutoML setting and typical optimization problems is illustrated.
    \item As the foundation for building an ensemble of optimizers a selection of established optimization methods is given and explained.
\end{enumerate}
The overview of optimization methods is concluded with the discussion of a theoretical drawback of using a single optimization method.
This discussion of a possible disadvantage is used as a starting point for the line of reasoning why an ensemble of optimizers is an approach to counteract this drawback.
Before explaining the ensemble approach in the next chapter, this line of reasoning is continued with a selection of related works, where other approaches that addressed this theoretical disadvantage of using a single optimization method are mentioned. 

\section{Classical Machine Learning}
\label{sec:theory:ml}
If a human is given a task where the correct solution or reaction is not evident, a humans has always the possibility to react with a random solution or with an arbitrary reaction.
But if the human has any prior first- or second-hand experience with the same of a similar task, the human can choose the reaction based on the memories of different outcomes for different reactions for the more or less similar prior task.
With the high abstraction level and very symbolic nature of human thinking and memorization, it is comparably easy for humans to recognize even remote similarities between tasks.\newline
This is very challenging for a computer in contrast, because the models of tasks, experiences, and outcomes of reactions to tasks have to be readable and understandable for a machine, i.e. be in any kind of structured and consistent format.
This setting can be formalized as $(T, P, E)$, where $T$ is a class of tasks, $P$ a performance measurement for solutions of a specific instances of the tasks class $T$ and $E$ is either given or collected experience, i.e. performance measurements for certain solutions in the context of specific task instance $t\in T$~\cite{Mitchell-MachineLearning}.\newline
Of course it is possible for a human programmer to manually specify the solution with the best performance measurement for any possible $t\in T$ but for a high $|T|$ this is rarely possible and viable.
Here, machine learning has its use-case: "Machine learning enables us to tackle tasks that are too difficult to solve with fixed programs written and designed by human beings"~\cite{Goodfellow-DeepLearning}.\newline
A high number of different types of task classes $T$ is imaginable, but one of the common one is \textit{Supervised Learning}.
In supervised learning, $T$ includes a fix set of all possible solutions $S$.
The concrete task for Supervised Learning is now to select for a given $t\in T$ a $s\subseteq S$ such that $P(t,s)$ is optimal.
To enable a machine to learn supervised, the experience $E$ has to be successively build in the form of ${(t_1, s_1, P(t_1, s_1)), ..., (t_n, s_n, P(t_n s_n))}$.
For a new task instance $t_i$, the computer will select a $s_i$ based on a decision model build from $E$, receive a performance feedback $P(t_i, s_i)$ and finally enrich $E$ with $(t_i, s_i, P(t_i, s_i))$ as well as updating the decision model with the changed $E$.
Therewith, a well working machine learning algorithm for supervised learning would now be able to achieve $P(t_j, s_j) \geq P(t_i, s_i)$ if $t_j$ and $t_i$ are similar instances, since it already has experience which $s\subseteq S$ had a certain performance value for $t_i$ and might therefore be a good or bad choice for $t_j$.\newline
This task of comparing and judging the similarity of different instances of $T$ and to build a practicable decision model based on $E$ to select a solution out of $S$ has been tackled with a plethora of different algorithms.
Usually, this algorithms have to be configured with a set of hyperparameter depending on $T$ and often $T$ also has to be transformed before presenting concrete instances to the algorithm for learning, i.e. pre-processing each $t\in T$ with one or more transformation methods.
Without making suitable choices for the machine learning algorithm as well as corresponding hyperparameter and pre-processing methods for each different $T$, the performance measurements will only increase for a high $|E|$ or even not at all.
But because the number of available task instances, i.e. the amount of data, that can be used to build $E$ with a machine learning algorithm is often limited for most use-case domains, a valid choice for the learning algorithm, hyperparameter and, if necessary, pre-processing methods is crucial.
With the high number of available machine learning algorithms and the often complex relationships between learning performance and hyperparameter configurations of the chosen algorithm as well as properties of $T$ and suitable pre-processing methods a big expertise in the machine learning field is necessary to be able to assemble machine learning applications with good performance measurements.

\section{Automated Machine Learning}
\label{sec:theory:automl}

\Blindtext

\subsection{Formalization of the AutoML Problem Setting}
\label{sec:theory:automl:workflow}

\Blindtext

\subsection{General Workflow}
\label{sec:theory:automl:workflow}

\Blindtext

\subsection{Model Selection}
\label{sec:theory:automl:selection}

\Blindtext

\subsection{Model Configuration}
\label{sec:theory:automl:configuration}

\Blindtext

\subsection{AutoML as an Optimization Problem}
\label{sec:theory:automl:optimization}

\Blindtext


\section{Black Box Optimization}
\label{sec:theory:optimization}

\Blindtext

\subsection{Differences to General Optimization Problems}
\label{sec:theory:optimization:differences}

\Blindtext

\subsection{Optimization by Searching}
\label{sec:theory:optimization:search}

\Blindtext

\subsection{Genetic and Evolutionary Optimization}
\label{sec:theory:optimization:genetic}

\Blindtext

\subsection{Bayesian Optimization}
\label{sec:theory:optimization:bayesian}

\Blindtext

\subsection{No-Free-Lunch Theorem}
\label{sec:theory:optimization:lunch}

\Blindtext

\section{Related Work}
\label{sec:theory:related}

\Blindtext
