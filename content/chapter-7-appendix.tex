% !TEX root = ../my-thesis.tex
%
\chapter{Example Appendix}
\label{sec:appendix}

\Blindtext[1][1]

\section{Example AutoML Optimization Space Definition}
\label{sec:appendix:htn-space}
This is an adaptation of the model selection and model configuration space of \textit{auto-sklearn}~\cite{Feurer-AutoSklearn} for the approach of this thesis.
It is modelled in the JSON format described in~\ref{sec:implementation:json} for modelling AutoML HTN planning spaces.\newline
Nevertheless, this modelling method is not capable of an exact one-to-one adaptation of the auto-sklearn space because their modelling given as Python code which is as a programming language more expressive than a data interchange language as JSON.
With this modelling in Python, it is possible to create constraints and relationships between parameters of a component.
For example something like if parameter $p_1$ has value $x$, $p_2$ cannot have value $y$, or $p_4$ will only be part of the model configuration if $p_3$ has value $z$.
Additionally, it is possible to include logical processing steps for the selected values after the model configuration in the Python code, for example to change a configured value depending on properties of the actual input dataset.\newline
Improving the modelling capacities of this JSON format can be a starting point for further research.
However, it will not be possible to make it as expressive as the model configuration space definitions in a turing-complete programming language as Python.

\Blindtext[1][1]

\section{Ensemble Optimizer AutoML in Pseudo-Code}
\label{sec:appendix:pseudo-code}

\Blindtext[1][1]

\section{Setup and Singularity Definition File of the Experiments}
\label{sec:appendix:singularity}
The evaluation utilizes a Python library for experiment execution.
It requires a definition of all possible input parameters for one experiment run and the desired result values.\newline
Experiment executions themselves, are orchestrated via one central database.
The experiment runner will select one possible parametrization of the allowed input parameter values, check via the database if this parametrization was already evaluated and perform the evaluation of the parametrization otherwise.
With this experiment management via a single database, the actual experiment execution can be parallelized arbitrarily.
Further details about the libraries setup and functionality can be found in the corresponding repository\footnote{\url{https://github.com/Berberer/python-experimenter}}.

The experiment runner has to be configured with the information regarding the database connection and the values for possible parametrization and expected results.
For example in the case of the experiments with the timeout of one hour to compare the different approaches regarding their accuracy scores and additionally to count the optimizer runs of frankenstein, the experiment configuration file could look like the following:
\begin{verbatim}
keyfields = timeout,seed,dataset,algorithm
resultfields = score,random_count,hyperband_count,genetic_count,
    smac_count,discretization_count

# Keyfield values(i.e. values for input parametrizations)
timeout=3600
seed=1,2,3,4,5,6,7,8,9,10,11,12,13,14,15,16,17,18,19,20,21,22,23,24,25,
    26,27,28,29,30
algorithm=tpot,autosklearn,mosaic,frankenstein
dataset=amazon-1000,car-6,cifar-3072,dexter-20000,dorothea-100000,
    krvskp-36,semeion-256,waveform-40,wine-11,yeast-8

# Database connection setup values
db.host = 192.168.0.1
db.type = MYSQL
db.username = experimenter
db.password = password123
db.database = experiments
db.table = automl
\end{verbatim}
Here, the datasets have a number attached to their name, to indicate the index of the column with the prediction target class.

This experiment configuration file is given as an input to the following Python script that defines how the experiments for a single parametrization are actually conducted and then starts the execution of experiments:
\begin{lstlisting}[language=Python]
import resource
from python_experimenter.experiment_runner import ExperimentRunner
from tpot import TPOTClassifier
from autosklearn.classification import AutoSklearnClassifier
from mosaic_ml.automl import AutoML
from sklearn.model_selection import train_test_split
from sklearn.metrics import accuracy_score
from frankensteins_automl.frankensteins_automl import (
    FrankensteinsAutoMLConfig,
    FrankensteinsAutoML,
)
from frankensteins_automl.machine_learning.arff_reader import read_arff
from frankensteins_automl.optimizers.baysian.smac_optimizer import SMAC


def automl_experiment(keyfields):
    # Get experiment parametrization
    algorithm = keyfields["algorithm"]
    dataset = keyfields["dataset"]
    seed = int(keyfields["seed"])
    timeout = int(keyfields["timeout"])

    # Set results to dummy values
    results = {}
    results["random_count"] = -1
    results["hyperband_count"] = -1
    results["genetic_count"] = -1
    results["smac_count"] = -1
    results["discretization_count"] = -1
    results["score"] = -1

    # Read input dataset and perform stratified split
    parts = dataset.split("-")
    dataset_file = parts[0]
    class_index = int(parts[1])
    data_x, data_y, _, _ = read_arff(
        f"res/datasets/{dataset_file}.arff",
        class_index 
    )
    x_train, x_test, y_train, y_test = train_test_split(
                data_x,
                data_y,
                test_size=0.3,
                random_state=seed,
                stratify=data_y,
            )
    score = None

    # Configure and execute approach of this experiments run
    if algorithm == "tpot":
        tpot_automl = TPOTClassifier(
            generations=None,
            random_state=seed,
            max_time_mins=int(timeout / 60.0)
        )
        tpot_automl.fit(x_train, y_train)
        score = tpot_automl.score(x_test, y_test)
    elif algorithm == "autosklearn":
        autosklearn_automl = AutoSklearnClassifier(
            time_left_for_this_task=timeout,
            seed=seed,
            ml_memory_limit=16384
        )
        autosklearn_automl.fit(x_train, y_train)
        predictions = autosklearn_automl.predict(x_test)
        score = accuracy_score(predictions, y_test)
    elif algorithm == "mosaic":
        mosaic_automl = AutoML(
            time_budget=timeout,
            seed=seed,
            scoring_func="accuracy",
            memory_limit=59392,
        )
        _, score = mosaic_automl.fit(
            x_train,
            y_train,
            x_test,
            y_test,
            categorical_features=["numerical"]*len(x_train[0])
        )
    else:
        config = FrankensteinsAutoMLConfig()
        config.direct_data_input(x_train, y_train)
        config.timeout_in_seconds = float(timeout)
        config.timeout_for_optimizers_in_seconds = 180.0
        config.timeout_for_pipeline_evaluation = 300.0
        config.simulation_runs_amount = 3
        config.random_seed = seed
        
        # Select frankensteins-automl variant
        if algorithm == "frankenstein":
            config.count_optimizer_calls = True
        elif algorithm == "frankenstein-rs":
            config.random_node_selection = True
        elif algorithm == "frankenstein-mcts":
            config.optimizers = [SMAC]

        # Return accuracy and optimizer run counts if available
        e = None
        r = None
        try:
            frankensteins_automl = FrankensteinsAutoML(config)
            r = frankensteins_automl.run()
        except Exception as exception:
            e = exception
        if r is not None:
            pipeline = r["pipeline_object"]
            if pipeline is not None:
                pipeline.fit(x_train, y_train)
                predictions = pipeline.predict(x_test)
                score = accuracy_score(predictions, y_test)
            if algorithm == "frankenstein":
                if "optimizer_calls" in r:
                    oc = r["optimizer_calls"]
                    if "RandomSearch" in oc:
                        results["random_count"] = oc["RandomSearch"]
                    if "Hyperband" in oc:
                        results["hyperband_count"] = oc["Hyperband"]
                    if "GeneticAlgorithm" in oc:
                        results["genetic_count"] = oc["GeneticAlgorithm"]
                    if "SMAC" in oc:
                        results["smac_count"] = oc["SMAC"]
                    if "DiscretizationSearch" in oc:
                        results["discretization_count"] = (
                            oc["DiscretizationSearch"]
                        )
        elif e is not None:
            raise e

    if score is not None:
        results["score"] = score

    return results

# Start the experiment execution with the here defined evaluation method
runner = ExperimentRunner(automl_experiment, "experiment.properties")
runner.run()
\end{lstlisting}
This experiment script can be used for the experiments of all research questions of this thesis with an appropriate configuration in the corresponding \texttt{experiment.properties} file.

At last, this script has to be executed in a reproducible Python environment.
For instance, this can be achieved with Singularity containers~\cite{Kurtzer-Singularity}.
The Singularity recipe for the container of this experiments is the following:
\begin{verbatim}
BootStrap: library
From: ubuntu:20.04

%files
    # Approach of this thesis to be copied into the container
    frankensteins-automl /experiment/

%post
    # Bring Ubuntu up to date
    apt-get -y update
    apt-get -y upgrade
    # Install Python and different required libraries
    apt-get -y install python3
    apt-get -y install python3-dev
    apt-get -y install python3-distutils
    apt-get -y install wget
    apt-get -y install git
    apt-get -y install bison
    apt-get -y install flex
    apt-get -y install curl
    apt-get -y install build-essential
    apt-get -y install autotools-dev
    apt-get -y install automake
    apt-get -y install libpcre3-dev
    apt-get -y install gcc
    apt-get -y install g++
    # Install pip
    wget https://bootstrap.pypa.io/get-pip.py
    python3 get-pip.py
    pip install --upgrade pip setuptools wheel Cython numpy scipy
    # Build and install swig
    git clone https://github.com/swig/swig.git --branch=v3.0.8
    cd swig
    ./autogen.sh
    ./configure
    make
    make install
    # Install approach of this thesis locally in the container
    cd /experiment/frankensteins-automl
    pip install -r requirements.txt
    pip install .
    cd ..
    # Clone and install experiment runner locally in the container
    git clone https://github.com/Berberer/python-experimenter.git
    cd python-experimenter
    pip install -r requirements.txt
    pip install .
    cd ..
    # Install the benchmark approaches and their dependencies
    pip install tpot
    a="https://raw.githubusercontent.com/automl/auto-sklearn/master/requirements.txt"
    curl ${a} | xargs -n 1 -L 1 pip install
    pip install auto-sklearn
    pip install git+https://github.com/herilalaina/mosaic@0.1
    pip install git+https://github.com/herilalaina/mosaic_ml
\end{verbatim}
This recipe assumes that the image is build in a directory where the reference implementation of the approach of this thesis is located in a folder called \texttt{frankensteins-automl} to allow direct changes and customizations of the code.
Alternatively, this can be easily changed in the recipe to directly clone it from GitHub.\newline
After building the image, the experiments can be started for example via \texttt{singularity exec experiment.sif python3 experiment.py}.
